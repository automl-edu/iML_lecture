\documentclass[aspectratio=169]{../latex_main/tntbeamer}


\input{../latex_main/preamble}

\title{iML: Post-hoc Methods for Neural Networks}
\subtitle{Visualizing Neural Networks}
\begin{document}
	\maketitle
	\graphicspath{ {./figure/} }
	
\begin{frame}[c]{Inspecting the Model units}
    \begin{itemize}
        \item Neural Networks architectural units can be inspected to provide insights
        \item What happens to the input signal as it travels through the network ?
        \begin{itemize}
            \item Activations: Activation in neural networks are sparse
            \item Attention units: Encode the importance of input representation units
        \end{itemize}
    \end{itemize}
    
    \centering
    \includegraphics[scale=1]{fertig}
\end{frame}

\begin{frame}{Visualizing Neural Network Architectural Units}
    \begin{itemize}
        \item Search for examples where individual features have high values —
        \begin{itemize}
            \item Either for a neuron at an individual position, or for an entire channel
        \end{itemize}
    \end{itemize}
    
    \includegraphics[scale=.47]{img180}
\end{frame}
    
\begin{frame}{Visualizing Filters in a CNN}
    \begin{itemize}
        \item Most of the aggregated values at neurons do not result in activations
        \item Find image patches in dataset that maximally activate/excite a unit
    \end{itemize}
    
\end{frame}

	
\begin{frame}{Feature extraction evolution}
    \begin{itemize}
        \item Lower layers extract lower-level features
        \item Higher layers compose extracted features to compose high-level features
    \end{itemize}
    \begin{figure}[h]
         \includegraphics[scale=.5]{img241}
         \centring
         \includegraphics[scale=.5]{img245}
         \hfill
         \includegraphics[scale=.4]{img249}
    \end{figure}
\end{frame}
	
	
	
	
	
	
	
	
	
\end{document}	